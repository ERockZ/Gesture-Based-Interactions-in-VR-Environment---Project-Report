\section{Conclusion}
%This is the conclusion, pointing out the main learnings from our results.
The results generated in this study draw an ambivalent picture about the suitability of the two chosen methods for system interaction in VR environments at this point in time.
On the one hand, many participants explicitly stated they preferred using the LEAP for carrying out the presented placement task within a VR environment.
On the other hand overall results regarding accuracy \& SUS usability score support the claim that a better overall experience was possible under the Android condition (even more so considering the increased difficulties experienced under this condition).
Perhaps the novelty aspect of the LEAP interaction, the fact that participants' hands find a direct manifestation within the VR scene and the arguably much higher immersion that is facilitated hereby contribute to the popularity of the LEAP system amongst the participants, outweighing the factual evidence that a more precise and smoother (regarding the gesture recognition) object manipulation was experienced under the Android condition.

As sensors like the LEAP and their underlying algorithms for modeling and recognizing hand postures advance in development they almost certainly will enable a much enhanced experience and eventually catch up with more conservative means of systems interaction such as touch gestures recognition.